W ramach projektu przewidziane zostały eksperymenty:
\begin{itemize}
    \item badanie wpływu ilości drzew, składających się na las losowy na jakość klasyfikacji (w tym las składający się z jednego drzewa);
    \item ilości zmiennych losowanych w każdym węźle - weryfikacja tezy, że najkorzystniejsze jest wykorzystywanie $\sqrt {ilosc atrybutow}$;
    \item maksymalna głębokości pojedynczego drzewa;
    \item wpływ losowania przypadków do modelu ze zwracaniem i bez;
    \item badanie predykcji dla danych z niezbalansowanymi klasami;
        \item wprowadzenie wag danych klas;
        \item dla naiwnego klasyfikatora bayesowskiego spawdzony zostanie wpływ wygladzenia Laplace'a;
        
\end{itemize}
