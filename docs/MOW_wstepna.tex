\documentclass{article}
\pdfpagewidth=8.5in
\pdfpageheight=11in

\usepackage{MOWreport}
% Use the postscript times font!
\usepackage{times}
\usepackage{soul}
\usepackage{url}
\usepackage{xcolor}
\usepackage{polski}
\usepackage[polish]{babel}
\usepackage[utf8]{inputenc}
\usepackage[T1]{fontenc}
\usepackage[utf8]{luainputenc}
\usepackage[hidelinks]{hyperref}
\usepackage[utf8]{inputenc}
\usepackage{caption}
\usepackage{indentfirst}
\usepackage{graphicx}
\usepackage{amsmath}
\usepackage{siunitx}
\usepackage{booktabs}
\usepackage{placeins}
\usepackage{pdfpages}
\usepackage{subfig}
\urlstyle{same}

\title{{\normalfont Metody Odkrywania Wiedzy}\vspace{0.2cm} \\ 
Przewidywanie spożycia alkoholu przez studentów przy użyciu lasu losowego\\
\vspace{0.2cm}
\large{\normalfont Dokumentacja wstępna}}

\author{
Joanna Kiesiak, Julia Kłos\\

}


\newcommand{\todo}[1]{\textcolor{blue}{\textbf{TO DO:} #1}}

\begin{document}

\maketitle

\section{Opis projektu}
Celem projektu jest zbadanie własności algorytmu lasu losowego. Dodatkowo należy porównać wyniki klasyfikacji z~wynikami uzyskanymi przy użyciu naiwnego klasyfikatora bayesowskiego.

Jednym z łatwiejszych do zrozumienia, wizualizacji oraz interpretacji algorytmów jest drzewo decyzyjne. Umożliwia ono wykorzystanie zmiennych zarówno numerycznych, jak i
kategorycznych, a zastosowanie znajduje w zadaniach zarówno klasyfikacyjnych, jak i regresyjnych. Niestety, charakteryzuje je również kilka istotnych wad, takich jak: niestabilność (kiedy niewielka zmiana w danych może doprowadzić do zupełnie innego wyniku
końcowego) oraz nadmierne dopasowanie.

Rozwiązaniem tych problemów są procedury agregujące rodziny klasyfikatorów, np. lasy losowe.
Działanie jego polega na klasyfikacji za pomocą grupy drzew decyzyjnych. Każde z~drzew wskazuje
ocenę odpowiednią dla danego przypadku, a~wynik końcowy jest efektem głosowania większościowego lub średnią poszczególnych przewidywań.

Algorytm lasu losowego zostanie zbadany i wykorzystany do przewidywania spożycia alkoholu przez studentów. Badana grupa zostanie podzielona według częstotliwości spożycia alkoholu (w skali 1--5, gdzie 1 to bardzo rzadko, a 5 bardzo często). Oddzielone zostało spożywanie alkoholu w trakcie tygodnia od weekendów.
\section{Opis algorytmów}
\subsection{Drzewo decyzyjne}
Modele drzew mogą zostać wykorzystane do zagadnień klasyfikacyjnych jak i regresyjnych.
Rekurencyjna struktura drzewa składa się z:
\begin{itemize}
    \item węzłów, które reprezentują testy na wartościach atrybutu;
    \item gałęzi, oznaczających możliwe wyniki testu dla danego węzła;
    \item liści, czyli decyzji o klasyfikacji obiektu.
\end{itemize}
Proces predykcji polega na przejściu od pierwszego węzła do pewnego liścia, ścieżką wyznaczaną przez gałęzie odpowiadające wartościom atrybutów danego przykładu.

Budowa drzewa wiążę się z wieloma ważnymi decyzjami, m.in. : czy utworzyć węzeł czy liść,  jakiego podziału dokonać w węźle, którą klasę będzie reprezentował dany liść. Rozbudowa drzewa zostaje wstrzymana zazwyczaj wtedy, gdy osiągnie ono maksymalną (ustaloną wcześniej) głębokość, bądź gdy zbiór obiektów przeznaczonych do budowania modelu jest już pusty lub jednorodny. Wyznaczanie etykiety danego liścia odbywa się zazwyczaj metodą większościową, tzn.  dla danego zbioru obiektów wybierana jest klasa
decyzyjna najliczniej reprezentowana w tym zbiorze

Spośród możliwych modeli, preferowane są drzewa o prostszej budowie, dlatego dąży się do jak najszybszego uzyskania jednorodnych klas. Oznacza to, że podział aktualnego zbioru przykładów ma dać jak największy przyrost informacji, uzyskać najmniejszą nieczystość
klas po podziale. 
\todo{entropia, gini, przycinanie... ogólne cechy drzew}
\subsection{Las losowy}
Działanie lasów losowych polega na klasyfikacji za pomocą grupy drzew decyzyjnych. Parametrem, który odpowiada za finalną decyzję jest średnia, gdy przewidywana jest wartość liczbowa lub wynik głosowania dla analizowanej przynależności do klasy.

Każde z drzew w lasie losowym jest tworzone w oparciu o próbę, powstałą przez wylosowanie N obiektów ze zbioru uczącego. W każdym węźle danego drzewa podział jest dokonywany na podstawie części losowo wybranych cech, których liczba jest zazwyczaj mniejsza od liczby wszystkich cech. Ma to pozwolić na uzyskanie jak największej niezależności poszczególnych węzłów, czyli zmniejszenie wariancji modelu. 
Błąd klasyfikacji może być szacowany na podstawie obiektów nie włączonych do próby.

Wykorzystanie lasu losowego ma zmniejszyć błąd klasyfikacji, poprzez zmniejszenie wpływu pojedynczego, nieidealnego klasyfikatora na końcową decyzję, zachowując jednocześnie jego zalety. Wynika
to z założenia, że większość drzew zapewni dobre przewidywanie dla większości danych, a każde
drzewo myli się w innym miejscu. Często nie stosuje się operacji przycinania drzewa, która ma zmniejszyć podatność na dopasowanie, ponieważ ma temu przeciwdziałać struktura lasu losowego.  Ten algorytm jest wykorzystywany do analiz na dużych zbiorach danych,
a także brakujących.






\subsection{Naiwny klasyfikator bayesowski}
Naiwny klasyfikator bayesowski jest modelem probabilistycznym opartym na klasycznym twierdzeniu Bayesa \cite{lewis1998naive}. Opisuje ono
relację pomiędzy prawdopodobieństwem warunkowym pewnego zdarzenia oraz jego prawdopodobieństwem bezwarunkowym:
\begin{equation}
\label{eq:bayes}
    P(A|B) = \frac{P(A)\times P(B|A)}{P(B)}.
\end{equation}
Wykorzystując wzór (\ref{eq:bayes}), można znaleźć prawdopodobieństwo zdarzenia \mathbf{A}, zakładając, że zdarzenie \mathbf{B} wystąpiło. Zdarzenie \mathbf{B} można potraktować jako dowód, a zdarzenie \mathbf{A} jako hipotezę.

Klasyfikator bayesowski przypisuje najbardziej prawdopodobną klasę $c_k$ dla danego przykładu na podstawie jej wektora atrybutów  $\pmb{x} = [a_1=v_1, a_2=v2,..., a_n = v_n]$ \cite{lewis1998naive}:
\begin{equation}
    \label{eq:bayes_class}
     P(c_k|\pmb{x}) = \frac{P(c_k)\times P(\pmb{x}|c_k)}{P(\pmb{x})}.
\end{equation}

Ten klasyfikator silnie korzysta z założenia o niezależności. Co oznacza, że prawdopodobieństwo żadnego z atrybutów nie wpływa na prawdopodobieństwo innego z atrybutów \cite{mukherjee2012intrusion}:
\begin{equation}
   P(a_1=v_1, a_2=v2,..., a_n = v_n|c_k) = \prod_{i=1}^{n}{P(a_i=v_i|c_k)}}
\end{equation}

Algorytm charakteryzuje prostota, łatwość implementacji i~niska złożoność obliczeniowa. Jest odporny na dopasowanie i wydaje się być najbardziej skuteczny, przy zbiorach danych o dużej liczbie atrybutów.
\section{Plan badań}
\subsection{Charakterystyka zbioru danych}
\todo{+ewentualnych czynności związanych z przygotowaniem danych}
Na dane składają się wyniki ankiet zebranych pośród uczestników lekcji matematyki (395 osób) i hiszpańskiego (649 osób). Część osób (392) uzupełniło ankiety w obu grupach, dlatego należy je rozpoznać na podstawie niektórych atrybutów (informacji o szkole, miejscu zamieszkania, płci, danych o rodzinie) i usunąć ze zbioru danych. W ten sposób uzyskano 622 unikatowych wyników ankiety.

Dane składają się z 33 atrybutów (zestawienie w tab. \ref{tab:attributes}), z czego 2 dotyczą spożycia alkoholu, czyli klas, które mają zostać przewidziane przez algorytmy. 
\todo{skończyć}
\begin{table}[h]
\centering
\caption{Atrybuty analizowanych danych}
\label{tab:attributes}
\begin{tabular}{|c|c|c|}
\hline
Nazwa & Rodzaj & Opis \\ \hline
school   &      &    \\ \hline
sex   &      &    \\ \hline
age   &      &    \\ \hline
address   &      &    \\ \hline
famsize   &      &    \\ \hline
Pstatus   &      &    \\ \hline
Medu   &      &    \\ \hline
Fedu   &      &    \\ \hline
Mjob   &      &    \\ \hline
Fjob   &      &    \\ \hline
reason   &      &    \\ \hline
guardian   &      &    \\ \hline
traveltime   &      &    \\ \hline
studytime   &      &    \\ \hline
failures   &      &    \\ \hline
schoolsup   &      &    \\ \hline
famsup   &      &    \\ \hline
paid  &      &    \\ \hline
activities   &      &    \\ \hline
nursery   &      &    \\ \hline
higher   &      &    \\ \hline
internet   &      &    \\ \hline
romantic   &      &    \\ \hline
famrel  &      &    \\ \hline
freetime   &      &    \\ \hline
gouot   &      &    \\ \hline
\textbf{Dalc}  &      &    \\ \hline
\textbf{Walc}   &      &    \\ \hline
health   &      &    \\ \hline
absences  &      &    \\ \hline
G1  &      &    \\ \hline
G2  &      &    \\ \hline
G3   &      &    \\ \hline
\end{tabular}
\end{table}
\subsection{Cel poszczególnych eksperymentów}
\todo{pytania, na które będzie poszukiwana odpowiedź, lub hipotezy do weryfikacji}
Celem badań jest sprawdzenie wpływu poszczególnych parametrów na  jakość klasyfikacji. Porównana zostanie skuteczność pojedynczego drzewa z grupą takich modeli, czyli lasem losowym. Dodatkowo zweryfikowana zostanie hipoteza jakoby las losowy radził sobie z jedną z największych wad drzewa decyzyjnego, czyli nadmiernym dopasowaniem do danych.

Porównana zostanie również predykcja dla danych z niezbalansowanymi klasami. Wszystkie wyniki klasyfikacji zostaną zestawione z wynikami naiwnego klasyfikatora bayesowskiego, również stosunkowo prostego modelu.

Celem badań jest porównanie dwóch modeli predykcyjnych - jeden model zbudowany z lasu losowego oraz drugi model zbudowany z klasyfikatora bayesowskiego. W tym celu zbiór danych zostanie podzielony na zbiór treningowy oraz testowy. Na zbiorze treningowym zostanie wykonane uczenie wspomnianych modeli, a na zbiorze testowym będziemy sprawdzać jak bardzo uzyskany wynik różni się od wartości rzeczywistej.

W eksperymencie odpowiemy na pytania: czy liczba atrybutów ma wpływ na jakość modelu, czy bootstraping polepsza a może pogarsza predykcję, czy większa liczba drzew powoduje lepszą predykcję. Uzyskane wyniki zostaną zebrane w tabeli oraz podsumowane.
\subsection{Badane parametry algorytmów}
W ramach projektu przewidziane zostały eksperymenty:
\begin{itemize}
    \item badanie wpływu ilości drzew, składających się na las losowy, na jakość klasyfikacji (w tym las składający się z jednego drzewa);
    \item ilości zmiennych losowanych w każdym węźle (weryfikacja tezy, że najkorzystniejsze jest wykorzystywanie $\sqrt {ilość atrybutów}$;
    \item maksymalna głębokości pojedynczego drzewa;
    \item wypływ losowania przypadków do modelu ze zwracaniem i bez;
    \item badanie predykcji dla danych z niezbalansowanymi klasami;
        \item wprowadzenie wag danych klas'
        \item dla naiwnego klasyfikatora bayesowskiego spawdzony zostanie wpływ wygladzenia Laplace'a;
        \item \todo{zgrabniej opisać, więcej?}
        
\end{itemize}


\subsection{Miary jakości i procedury oceny modeli}
Ocena modelu predykcyjnego odbędzie się na podstawie poniższych parametrów:
\begin{itemize}
    \item suma kwadratów różnicy wartości predykcyjnej od wartości rzeczywistej
    \begin{equation}
    MSE = {\frac{1}{n}\sum_{i=1}^{n}(Y_{i} - y_{i})^{2}}
    \end{equation}
    \item pierwiastek sumy kwadratów różnicy wartości predykcyjnej od wartości rzeczywistej
    \begin{equation}
    RMSE = \sqrt{(\frac{1}{n})\sum_{i=1}^{n}(Y_{i} - y_{i})^{2}}
    \end{equation}
    \item sprawdzenie confusion matrix - czyli informacja o klasyfikacji próbek fałszywie pozytywnych, prawdziwie fałszywych itd;
    \item sprawdzenie dokładności modelu przez użycie walidacji krzyżowej;
    \item w modelu zastosujemy bootstrapping, czyli K-krotne użycie pierwotnego datasetu do uczenia modelu, w tym wypadku zbadamy zagregowany błąd predykowany w agregacie do zagregowanego błędu rzeczywistego;
\end{itemize}

\todo{yghh}
\section{Otwarte kwestie}
\todo{czy konieczne bedzie dodatkowe przeciwdzialanie niezbalansowanym klasom?}
\bibliographystyle{abbrv}
\bibliography{MOWbib}
\end{document}