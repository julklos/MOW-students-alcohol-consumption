\documentclass{article}
\pdfpagewidth=8.5in
\pdfpageheight=11in

\usepackage{MOWreport}
% Use the postscript times font!
\usepackage{times}
\usepackage{soul}
\usepackage{url}
\usepackage{xcolor}
\usepackage{polski}
\usepackage[polish]{babel}
\usepackage[utf8]{inputenc}
\usepackage[T1]{fontenc}
\usepackage[utf8]{luainputenc}
\usepackage[hidelinks]{hyperref}
\usepackage[utf8]{inputenc}
\usepackage{caption}
\usepackage{indentfirst}
\usepackage{graphicx}
\usepackage{amsmath}
\usepackage{siunitx}
\usepackage{booktabs}
\usepackage{subfig}
\urlstyle{same}

\title{{\normalfont Metody Odkrywania Wiedzy}\vspace{0.2cm} \\ 
Przewidywanie spożycia alkoholu przez studentów przy użyciu lasu losowego\\
\vspace{0.2cm}
\large{\normalfont Dokumentacja wstępna}}

\author{
Joanna Kiesiak, Julia Kłos\\

}


\newcommand{\todo}[1]{\textcolor{blue}{\textbf{TO DO:} #1}}

\begin{document}

\maketitle

\section{Opis projektu}
 \todo{Szczegółowa interpretacja tematu projektu}
\section{Opis algorytmów}
\subsection{Drzewo}
\subsection{Las losowy}
\subsection{Naiwny klasyfikator bayesowski}

\section{Plan badań}
\subsection{Charakterystyka zbioru danych}
\todo{+ewentualnych czynności związanych z przygotowaniem danych}
\subsection{Cel poszczególnych eksperymentów}
\todo{pytania, na które będzie poszukiwana odpowiedź, lub hipotezy do weryfikacji}
\subsection{Badane parametry algorytmów}
\subsection{Miary jakości i procedury oceny modeli}
\section{Otwarte kwestie}
\bibliographystyle{abbrv}
\bibliography{MOWbib}
\end{document}