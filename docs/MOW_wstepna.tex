\documentclass{article}
\pdfpagewidth=8.5in
\pdfpageheight=11in

\usepackage{MOWreport}
% Use the postscript times font!
\usepackage{times}
\usepackage{soul}
\usepackage{url}
\usepackage{xcolor}
\usepackage{polski}
\usepackage[polish]{babel}
\usepackage[utf8]{inputenc}
\usepackage[T1]{fontenc}
\usepackage[utf8]{luainputenc}
\usepackage[hidelinks]{hyperref}
\usepackage[utf8]{inputenc}
\usepackage{caption}
\usepackage{indentfirst}
\usepackage{graphicx}
\usepackage{amsmath}
\usepackage{siunitx}
\usepackage{booktabs}
\usepackage{subfig}
\urlstyle{same}

\title{{\normalfont Metody Odkrywania Wiedzy}\vspace{0.2cm} \\ 
Przewidywanie spożycia alkoholu przez studentów przy użyciu lasu losowego\\
\vspace{0.2cm}
\large{\normalfont Dokumentacja wstępna}}

\author{
Joanna Kiesiak, Julia Kłos\\

}


\newcommand{\todo}[1]{\textcolor{blue}{\textbf{TO DO:} #1}}

\begin{document}

\maketitle

\section{Opis projektu}
Celem projektu jest zbadanie własności algorytmu lasu losowego. Dodatkowo należy porównać wyniki klasyfikacji z~wynikami uzyskanymi przy użyciu naiwnego klasyfikatora bayesowskiego.

Jednym z łatwiejszych do zrozumienia, wizualizacji oraz interpretacji algorytmów jest drzewo decyzyjne. Umożliwia ono wykorzystanie zmiennych zarówno numerycznych, jak i
kategorycznych, a zastosowanie znajduje w zadaniach zarówno klasyfikacyjnych, jak i regresyjnych. Niestety, charakteryzuje je również kilka istotnych wad, takich jak: niestabilność (kiedy niewielka zmiana w danych może doprowadzić do zupełnie innego wyniku
końcowego) oraz nadmierne dopasowanie.

Rozwiązaniem tych problemów są procedury agregujące rodziny klasyfikatorów, np. lasy losowe.
Działanie jego polega na klasyfikacji za pomocą grupy drzew decyzyjnych. Każde z~drzew wskazuje
klasę odpowiednią dla danego przypadku, a~wynik końcowy jest efektem głosowania większościowego.

Algorytm lasu losowego zostanie zbadany i wykorzystany do przewidywania spożycia alkoholu przez studentów. Badana grupa zostanie podzielona według częstotliwości spożycia alkoholu (w skali 1--5, gdzie 1 to bardzo rzadko, a 5 bardzo często). Oddzielone zostało spożywanie alkoholu w trakcie tygodnia od weekendów.
\section{Opis algorytmów}
\subsection{Drzewo}
\subsection{Las losowy}
\subsection{Naiwny klasyfikator bayesowski}

\section{Plan badań}
\subsection{Charakterystyka zbioru danych}
\todo{+ewentualnych czynności związanych z przygotowaniem danych}
\subsection{Cel poszczególnych eksperymentów}
\todo{pytania, na które będzie poszukiwana odpowiedź, lub hipotezy do weryfikacji}
\subsection{Badane parametry algorytmów}
\subsection{Miary jakości i procedury oceny modeli}
\section{Otwarte kwestie}
\bibliographystyle{abbrv}
\bibliography{MOWbib}
\end{document}