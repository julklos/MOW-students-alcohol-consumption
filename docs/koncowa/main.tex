\documentclass{article}

% If you're new to LaTeX, here's some short tutorials:
% https://www.overleaf.com/learn/latex/Learn_LaTeX_in_30_minutes
% https://en.wikibooks.org/wiki/LaTeX/Basics

% Formatting
\usepackage[utf8]{inputenc}
\usepackage[margin=1in]{geometry}
\usepackage[titletoc,title]{appendix}
\usepackage{graphicx}
\usepackage[T1]{fontenc}
\usepackage{libertine}
\usepackage{amsmath}

% Math
% https://www.overleaf.com/learn/latex/Mathematical_expressions
% https://en.wikibooks.org/wiki/LaTeX/Mathematics
\usepackage{amsmath,amsfonts,amssymb,mathtools}

% Images
% https://www.overleaf.com/learn/latex/Inserting_Images
% https://en.wikibooks.org/wiki/LaTeX/Floats,_Figures_and_Captions
\usepackage{graphicx,float}

% Tables
% https://www.overleaf.com/learn/latex/Tables
% https://en.wikibooks.org/wiki/LaTeX/Tables

% Algorithms
% https://www.overleaf.com/learn/latex/algorithms
% https://en.wikibooks.org/wiki/LaTeX/Algorithms
\usepackage[ruled,vlined]{algorithm2e}
\usepackage{algorithmic}
\usepackage{polski}
% Code syntax highlighting
% https://www.overleaf.com/learn/latex/Code_Highlighting_with_minted
\usepackage{minted}
\usemintedstyle{borland}

% References
% https://www.overleaf.com/learn/latex/Bibliography_management_in_LaTeX
% https://en.wikibooks.org/wiki/LaTeX/Bibliography_Management
\usepackage{biblatex}
\addbibresource{references.bib}

% Title content
\title{{\normalfont Metody Odkrywania Wiedzy}\vspace{0.2cm} \\ 
\textbf{Przewidywanie spożycia alkoholu przez studentów przy użyciu lasu losowego}\\
\vspace{0.2cm}
\large{\normalfont Dokumentacja końcowa}}

\author{
Joanna Kiesiak, Julia Kłos\\

}


\newcommand{\todo}[1]{\textcolor{blue}{\textbf{TO DO:} #1}}

\date{17 maja 2020 r.}

\begin{document}

\maketitle

% Abstract


% Introduction and Overview
\section{Opis projektu}
\todo{Krótki opis o o chodzi, jakie algorytmy itd.}

%  Theoretical Background
\section{Przeprowadzone badania}

\todo{nakreślenie co badałysmy itd.}

W ramach projektu przewidziane zostały eksperymenty:
\begin{itemize}
    \item badanie wpływu ilości drzew, składających się na las losowy na jakość klasyfikacji (w tym las składający się z jednego drzewa);
    \item ilości zmiennych losowanych w każdym węźle - weryfikacja tezy, że najkorzystniejsze jest wykorzystywanie $\sqrt {ilosc atrybutow}$;
    \item maksymalna głębokości pojedynczego drzewa;
    \item wpływ losowania przypadków do modelu ze zwracaniem i bez;
    \item badanie predykcji dla danych z niezbalansowanymi klasami;
        \item wprowadzenie wag danych klas\footnote[*]{opcjonalne};
        \item dla naiwnego klasyfikatora bayesowskiego spawdzony zostanie wpływ wygladzenia Laplace'a;
        
\end{itemize}

Część danych (ok. 10-20\%) zostanie wydzielona jako zbiór testujący, pozostała część będzie stanowiła zbiór uczący. Pozwoli to ocenić, jak dobrze model radzi sobie z uogólnianiem pojęcia. 

Ocena modelu predykcyjnego odbędzie się na podstawie poniższych parametrów:
\begin{itemize}
    \item suma kwadratów różnicy wartości predykcyjnej od wartości rzeczywistej
    \begin{equation}
    MSE = {\frac{1}{n}\sum_{i=1}^{n}(Y_{i} - y_{i})^{2}}
    \end{equation}
    \item pierwiastek sumy kwadratów różnicy wartości predykcyjnej od wartości rzeczywistej
    \begin{equation}
    RMSE = \sqrt{(\frac{1}{n})\sum_{i=1}^{n}(Y_{i} - y_{i})^{2}}
    \end{equation}
    \item sprawdzenie confusion matrix - czyli informacja o klasyfikacji próbek fałszywie i prawdziwie sklasyfikowanych dla poszczególnych klas, co pozwoli ocenić jak model radzi sobie z niezbalansowanymi klasami;
    \item sprawdzenie dokładności modelu przez użycie walidacji krzyżowej;
    \item w modelu zastosujemy bootstrapping, czyli K-krotne użycie pierwotnego zbioru danych do uczenia modelu, w tym wypadku zbadamy zagregowany błąd predykowany w agregacie do zagregowanego błędu rzeczywistego;
\end{itemize}


% las losowy opi
\section{Las losowy}
Działanie lasów losowych polega na klasyfikacji za pomocą grupy drzew decyzyjnych. Parametrem, który odpowiada za finalną decyzję jest średnia, gdy przewidywana jest wartość liczbowa lub wynik głosowania dla analizowanej przynależności do klasy.

Każde z drzew w lasie losowym jest tworzone w oparciu o próbę, powstałą przez wylosowanie N obiektów ze zbioru uczącego. W każdym węźle danego drzewa podział jest dokonywany na podstawie części losowo wybranych cech, których liczba jest zazwyczaj mniejsza od liczby wszystkich cech. Ma to pozwolić na uzyskanie jak największej niezależności poszczególnych węzłów, czyli zmniejszenie wariancji modelu. 
Błąd klasyfikacji może być szacowany na podstawie obiektów nie włączonych do próby.

Wykorzystanie lasu losowego ma zmniejszyć błąd klasyfikacji, poprzez zmniejszenie wpływu pojedynczego, nieidealnego klasyfikatora na końcową decyzję, zachowując jednocześnie jego zalety. Wynika
to z założenia, że większość drzew zapewni dobre przewidywanie dla większości danych, a każde
drzewo myli się w innym miejscu. Często nie stosuje się operacji przycinania drzewa, która ma zmniejszyć podatność na dopasowanie, ponieważ ma temu przeciwdziałać struktura lasu losowego.  Ten algorytm jest wykorzystywany do analiz na dużych zbiorach danych,
a także brakujących.







\section{Klasyfikator Bayesowski}
Naiwny klasyfikator bayesowski jest modelem probabilistycznym opartym na klasycznym twierdzeniu Bayesa \cite{lewis1998naive}. Opisuje ono
relację pomiędzy prawdopodobieństwem warunkowym pewnego zdarzenia oraz jego prawdopodobieństwem bezwarunkowym:
\begin{equation}
\label{eq:bayes}
    P(A|B) = \frac{P(A)\times P(B|A)}{P(B)}.
\end{equation}
Wykorzystując wzór (\ref{eq:bayes}), można znaleźć prawdopodobieństwo zdarzenia \mathbf{A}, zakładając, że zdarzenie \mathbf{B} wystąpiło. Zdarzenie \mathbf{B} można potraktować jako dowód, a zdarzenie \mathbf{A} jako hipotezę.

Klasyfikator bayesowski przypisuje najbardziej prawdopodobną klasę $c_k$ dla danego przykładu na podstawie jej wektora atrybutów  $\pmb{x} = [a_1=v_1, a_2=v2,..., a_n = v_n]$ \cite{lewis1998naive}:
\begin{equation}
    \label{eq:bayes_class}
     P(c_k|\pmb{x}) = \frac{P(c_k)\times P(\pmb{x}|c_k)}{P(\pmb{x})}.
\end{equation}

Ten klasyfikator silnie korzysta z założenia o niezależności. Co oznacza, że prawdopodobieństwo żadnego z atrybutów nie wpływa na prawdopodobieństwo innego z atrybutów \cite{mukherjee2012intrusion}:
\begin{equation}
   P(a_1=v_1, a_2=v2,..., a_n = v_n|c_k) = \prod_{i=1}^{n}{P(a_i=v_i|c_k)}}
\end{equation}

Algorytm charakteryzuje prostota, łatwość implementacji i~niska złożoność obliczeniowa. Jest odporny na dopasowanie i wydaje się być najbardziej skuteczny, przy zbiorach danych o dużej liczbie atrybutów.

% Summary and Conclusions
\section{Podsumowanie}

W ramach projektu przewidywania spożycia alkoholu przez studentów przy użyciu lasu losowego oraz klasyfikatora Bayesa przeprowadzono szereg badań, aby osiągnąć jak największą predykcję. Utrudnieniem w procesie modelowania było niezbalansowanie klas, które miało znaczący wpływ na odpowiednie dobranie parametrów oraz ostateczną predykcję.

W przypadku lasu losowego zbadano który parametr zwiększa detekcję, czy wpływ na model ma losowanie ze zwracaniem próbek. Udało się ustalić że predykcja w przypadku spożywania alkoholu w weekend jest wyższa niż w tygodniu. Jest to spodziewany wynik, ponieważ więcej studentów spożywa alkohol w weekend. Lepsze wyniki dla poszczególnych klas, bardziej równomierny rozkład błędu uzyskano przy zastosowaniu naiwnego klasyfikatora bayesowskiego. Znacząco lepiej poradził sobie z rozdzielaniem klas, podczas gdy las losowy preferował najsilniej reprezentowaną grupę.

W obu podejściach klasyfikacyjnych problem z którym się spotkano - to znacząca klasyfikacja próbek do niepoprawnych klas, w większość przypisywanie próbki do klasy pierwszej, czyli klasy która miała największy udział proporcji w zestawie danych. Wskazuje to jak istotne jest zebranie reprezentatywnych danych.

% References
\printbibliography

% Appendices
\begin{appendices}


\

\end{appendices}

\end{document}
