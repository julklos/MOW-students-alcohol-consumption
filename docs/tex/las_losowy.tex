Działanie lasów losowych polega na klasyfikacji za pomocą grupy drzew decyzyjnych. Przewidywaniem modelu jest średnia, gdy przewidywana jest wartość liczbowa bądź wynik głosowania, dla analizowanej przynależności do klasy. Wykorzystanie lasu losowego ma zmniejszyć błąd klasyfikacji, poprzez zmniejszenie wpływu pojedynczego, nieidealnego klasyfikatora na końcową decyzję, zachowując jednocześnie jego zalety. Wynika
to z założenia, że większość drzew zapewni dobre przewidywanie dla większości danych, a każde
drzewo myli się w innym miejscu. Ten algorytm ma umożliwić również pracę na dużych danych,
także brakujących.

Każde z drzew jest tworzone w oparciu o próbę, powstałą przez wylosowanie N obiektów ze zbioru uczącego. W każdym węźle danego drzewa podział jest dokonywany  na podstawie części losowo wybranych cech, których liczba jest zazwyczaj mniejsza od liczby wszystkich cech. Ma to pozwolić na uzyskanie jak największej niezależności poszczególnych drzew, czyli zmniejszenie wariancji modelu. Błąd klasyfikacji może być szacowany na podstawie obiektów nie
włączonych do próby.



