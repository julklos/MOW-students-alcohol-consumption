W ramach projektu przewidywania spożycia alkoholu przez stundentów przy użyciu lasu losowego oraz klasyfikatora bayesa przeprowadzono szereg badań, aby osiągnąć jak największą predykcję. Utrudnieniem w procesie modelowania było niezbalansowanie klas, które miało znaczący wpływ na odpowiednie dobranie parametrów oraz ostateczną predykcję.

W przypadku lasu losowego zbadano który parametr zwiększa detekcję, czy wpływ na model ma losowanie ze zwracaniem próbek. Udało się ustalić że predykcja w przypadku spożywania alkoholu w weekend jest wyższa niż w tygodniu. Jest to normalna sytuacja, ponieważ więcej studentów spożywa alkohol w weekend.
\todo{bayes 3 zdania}
W obu podejściach klasyfikacyjncyh problem z którym się spotkano - to znacząca klasyfikacja próbek do niepoprawnych klas, w większość przypisywanie próbki do klasy pierwszej, czyli klasy która miała największy udział proporcji w datasecie.
