\todo{+ewentualnych czynności związanych z przygotowaniem danych}
Na dane składają się wyniki ankiet zebranych pośród uczestników lekcji matematyki (395 osób) i hiszpańskiego (649 osób). Część osób (392) uzupełniło ankiety w obu grupach, dlatego należy je rozpoznać na podstawie niektórych atrybutów (informacji o szkole, miejscu zamieszkania, płci, danych o rodzinie) i usunąć ze zbioru danych. W ten sposób uzyskano 622 unikatowych wyników ankiety.

Dane składają się z 33 atrybutów (zestawienie w tab. \ref{tab:attributes}), z czego 2 dotyczą spożycia alkoholu, czyli klas, które mają zostać przewidziane przez algorytmy. 
\todo{skończyć}
\begin{table}[h]
\centering
\caption{Atrybuty analizowanych danych}
\label{tab:attributes}
\begin{tabular}{|c|c|c|}
\hline
Nazwa & Rodzaj & Opis \\ \hline
school   &      &    \\ \hline
sex   &      &    \\ \hline
age   &      &    \\ \hline
address   &      &    \\ \hline
famsize   &      &    \\ \hline
Pstatus   &      &    \\ \hline
Medu   &      &    \\ \hline
Fedu   &      &    \\ \hline
Mjob   &      &    \\ \hline
Fjob   &      &    \\ \hline
reason   &      &    \\ \hline
guardian   &      &    \\ \hline
traveltime   &      &    \\ \hline
studytime   &      &    \\ \hline
failures   &      &    \\ \hline
schoolsup   &      &    \\ \hline
famsup   &      &    \\ \hline
paid  &      &    \\ \hline
activities   &      &    \\ \hline
nursery   &      &    \\ \hline
higher   &      &    \\ \hline
internet   &      &    \\ \hline
romantic   &      &    \\ \hline
famrel  &      &    \\ \hline
freetime   &      &    \\ \hline
gouot   &      &    \\ \hline
\textbf{Dalc}  &      &    \\ \hline
\textbf{Walc}   &      &    \\ \hline
health   &      &    \\ \hline
absences  &      &    \\ \hline
G1  &      &    \\ \hline
G2  &      &    \\ \hline
G3   &      &    \\ \hline
\end{tabular}
\end{table}