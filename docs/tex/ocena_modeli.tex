Część danych (ok. 10-20\%) zostanie wydzielona jako zbiór testujący, pozostała część będzie stanowiła zbiór uczący. Pozwoli to ocenić, jak dobrze model radzi sobie z uogólnianiem pojęcia.

Ocena modelu predykcyjnego odbędzie się na podstawie poniższych parametrów:
\begin{itemize}
    \item suma kwadratów różnicy wartości predykcyjnej od wartości rzeczywistej
    \begin{equation}
    MSE = {\frac{1}{n}\sum_{i=1}^{n}(Y_{i} - y_{i})^{2}}
    \end{equation}
    \item pierwiastek sumy kwadratów różnicy wartości predykcyjnej od wartości rzeczywistej
    \begin{equation}
    RMSE = \sqrt{(\frac{1}{n})\sum_{i=1}^{n}(Y_{i} - y_{i})^{2}}
    \end{equation}
    \item sprawdzenie confusion matrix - czyli informacja o klasyfikacji próbek fałszywie pozytywnych, prawdziwie fałszywych itd;
    \item sprawdzenie dokładności modelu przez użycie walidacji krzyżowej;
    \item w modelu zastosujemy bootstrapping, czyli K-krotne użycie pierwotnego zbioru danych do uczenia modelu, w tym wypadku zbadamy zagregowany błąd predykowany w agregacie do zagregowanego błędu rzeczywistego;
\end{itemize}
