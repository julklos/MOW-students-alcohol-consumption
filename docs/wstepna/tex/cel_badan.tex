Celem badań jest sprawdzenie wpływu poszczególnych parametrów na  jakość klasyfikacji. Porównana zostanie skuteczność pojedynczego drzewa z grupą takich modeli, czyli lasem losowym. Dodatkowo zweryfikowana zostanie hipoteza jakoby las losowy radził sobie z jedną z największych wad drzewa decyzyjnego, czyli nadmiernym dopasowaniem do danych.

Porównana zostanie również predykcja dla danych z niezbalansowanymi klasami, co oznacza, że powstaną osobno modele klasyfikujące ze względu na oba atrybuty dot. spożywania alkoholu. Wszystkie wyniki klasyfikacji zostaną zestawione z wynikami naiwnego klasyfikatora bayesowskiego, również stosunkowo prostego modelu.

Celem badań jest porównanie dwóch modeli predykcyjnych - jeden model zbudowany z lasu losowego oraz drugi model zbudowany z klasyfikatora bayesowskiego. W tym celu zbiór danych zostanie podzielony na zbiór treningowy oraz testowy. Na zbiorze treningowym zostanie wykonane uczenie wspomnianych modeli, a na zbiorze testowym będziemy sprawdzać jak bardzo uzyskany wynik różni się od wartości rzeczywistej.

W eksperymencie odpowiemy na pytania: czy liczba atrybutów ma wpływ na jakość modelu, czy bootstraping polepsza a może pogarsza predykcję, czy większa liczba drzew powoduje lepszą predykcję. Uzyskane wyniki zostaną zebrane w tabeli oraz podsumowane.