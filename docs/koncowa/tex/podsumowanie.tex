W ramach projektu przewidywania spożycia alkoholu przez studentów przy użyciu lasu losowego oraz klasyfikatora Bayesa przeprowadzono szereg badań, aby osiągnąć jak największą predykcję. Utrudnieniem w procesie modelowania było niezbalansowanie klas, które miało znaczący wpływ na odpowiednie dobranie parametrów oraz ostateczną predykcję.

W przypadku lasu losowego zbadano który parametr zwiększa detekcję, czy wpływ na model ma losowanie ze zwracaniem próbek. Udało się ustalić że predykcja w przypadku spożywania alkoholu w weekend jest wyższa niż w tygodniu. Jest to spodziewany wynik, ponieważ więcej studentów spożywa alkohol w weekend. Lepsze wyniki dla poszczególnych klas, bardziej równomierny rozkład błędu uzyskano przy zastosowaniu naiwnego klasyfikatora bayesowskiego. Znacząco lepiej poradził sobie z rozdzielaniem klas, podczas gdy las losowy preferował najsilniej reprezentowaną grupę.

W obu podejściach klasyfikacyjnych problem z którym się spotkano - to znacząca klasyfikacja próbek do niepoprawnych klas, w większość przypisywanie próbki do klasy pierwszej, czyli klasy która miała największy udział proporcji w zestawie danych. Wskazuje to jak istotne jest zebranie reprezentatywnych danych.
